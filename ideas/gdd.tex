\documentclass[12pt]{article}

\title{Steel Purge Game Design Document}

\begin{document}

\maketitle{}

\section{Overview}


\subsection{Concept}

Smooth high mobility platformer game with various types of weapons, levels, enemies and bosses. 

\subsection{Genre(s)}

Run n' gun, platformer, sidescroller, action, RPG.

\subsection{Game Structure}

The game is a 2D sidescroller where the primary level type will start from right to left in the first half of the level objective unlike standard sidescrollers. The first half of the level is about carrying an explosive to a point. Along the way, there will be several obstacles and enemies posing a challenge to the player. After reaching the point, the player plant's the explosive and a timer starts. The player must now go back to the initial spawn point of the level within the bomb time limit. The level design will be two-way, meaning the first half will go from right to left and be more slow-paced as opposed to the second half which is more fast-paced. Since this is the primary level type, there will be more ways to play the levels, but this is the main selling point. 

\subsection{Look and Feel}

Summarize the visuals and the basic style of the game. Use similar products as example. 

\section{Gameplay}

\subsection{Objectives}

\subsubsection{Plant the Storm Fuse}

At the start of the level, the player is given an explosive item called the "Storm Fuse". This is where the game starts from the  opposite side of the map, a.k.a from right to left as opposed to the natural sidescroller starting point. The player must plant the Storm Fuse to complete a part of the level. 

\subsubsection{Evacuate from the Explosion}

After the player completes the "Plant the Storm Fuse", a timer for the bomb is set and the player must escape the level before the timer runs out. If the timer runs out, the level explodes and the player dies. Completing this objective means completing the level. 

\subsection{Progression Systems}

\subsubsection{Perks}

\subsubsection{Weapon Unlocks and Pick-ups}

\subsection{Play Flow}

Explain how the gameplay elements are connected to each other. Does it feel natural?

\subsection{Side-Challenge Structure}

The side-challenges exist to motivate the player to complete the level fully to achieve a higher rank. At the end of a level, the player is given information on their performance for each side-challenge. Completing all side-challenges should grant the player a reward that will help them on the journey, but I have yet to figure out if that should be perk-points or something else. 

\subsubsection{Kill All Enemies}

\subsubsection{Break your Time-Limit Record}

\subsubsection{Break your Evacuation Record}

\subsection{Puzzle Structure}

Every level has one or two puzzles that are not so obvious to find. Each puzzle will require a unique weapon ability, or two, to complete. This puzzle structure's presence is similar to that of The Breath of The Wild. 

\section{Game Mechanics}

This is a critical part of the design document. Emphasize this section when writing.

\subsection{Rules}

What is the player allowed to do? What are the limits to every mechanic?

\subsection{Interactions}

Show a general model of the game universe to define how things interact. 

\subsection{Physics}

Based on game genre and perspective (3D or 2D), define the basic physics structure.

\subsection{Economy}

Not strictly about currency, but about resource-management. For example, cooldowns on abilities or remaining healing items in your inventory. 

\subsection{Movement}

Explain how the player moves through the game based on controls and physics definition.

\subsection{Objects}

Explain the different objects in the world and the interactions with them. For example, objects you can pick up or hazards. 

\subsection{Combat}

Define the combat mechanics and their intended use against opponents and opponent moves. 

\subsection{Actions}

Explain all the possible actions the player is able to perform. This includes opening doors, picking up objects or attacks.

\subsection{Screen Flow}

Explain how the screen behaves when performing certain actions.

\subsection{Game Options that Affect Gameplay}

Describe all the possible options the player has and how the effect the gameplay further. 

\subsection{Replaying and Saving}

\subsection{Secrets and Easter Eggs}

\section{Story}

\subsection{Back Story}

\subsection{Plot Elements}

\subsection{Game Story Progression}

\subsection{Cut Scenes}

Descriptions about the actors included, the setting and a set of scripts.

\section{Game World}

\subsection{Look and Feel}

Define the visual and structural style of the game world. Is it a large island (open world)? Are there multiple worlds (Super Mario)?

\subsection{Areas (For Each)}

\subsubsection{Description and Physical Characteristics}

\subsubsection{Relations to the Rest of the World}

\section{Characters}

\subsection{Characters (For Each)}

\subsubsection{Back Story}

\subsubsection{Personality}

\subsubsection{Appearance}

\subsubsection{Abilities}

\subsubsection{Relevance to the Story}

\subsubsection{Relationships}

\subsection{AI and Enemies}

\subsection{Non-combat and Friendly Characters}

\section{Levels}

\subsection{Training Level}

\subsection{Levels (For Each)}

\subsubsection{Synopsis}

\subsubsection{Prerequisite Materials and How it's Provided}

\subsubsection{Objectives}

\subsubsection{Details of what Happens in the Level}

\paragraph{Map}

\paragraph{Critical Paths}

\paragraph{Important Encounters}

\section{Interface}

\subsection{Visual System}

\subsubsection{HUD}

\subsubsection{Menus}

\subsubsection{Camera Model}

\subsection{Control System}

\subsection{Audio}

\subsection{Art}

\subsection{Help System}

\end{document}