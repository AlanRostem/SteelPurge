\documentclass[12pt]{article}

\title{Steel Purge Game Design Document}

\begin{document}

\maketitle{}

\section{Overview}


\subsection{Concept}

Smooth high mobility platformer game with various types of guns, levels, enemies and puzzles. The game will feature multiple different objectives than the standard level completion of all platformers. 

\subsection{Genre(s)}

Run n' gun, platformer, sidescroller, action, RPG.

\subsection{Game Structure}

The game is a 2D sidescroller where the primary level type will start from right to left in the first half of the level objective unlike standard sidescrollers. The first half of the level is about carrying an explosive to a point. Along the way, there will be several obstacles and enemies posing a challenge to the player. After reaching the point, the player plant's the explosive and a timer starts. The player must now go back to the initial spawn point of the level within the bomb time limit. The level design will be two-way, meaning the first half will go from right to left and be more slow-paced as opposed to the second half which is more fast-paced. Since this is the primary level type, there will be more ways to play the levels, but this is the main selling point. 

\subsection{Look and Feel}

Summarize the visuals and the basic style of the game. Use similar products as example. 

\section{Gameplay}

\subsection{Objectives}

\subsubsection{Plant the Storm Fuse}

At the start of the level, the player is given an explosive item called the "Storm Fuse". This is where the game starts from the  opposite side of the map, a.k.a from right to left as opposed to the natural sidescroller starting point. The player must plant the Storm Fuse to complete a part of the level. 

\subsubsection{Evacuate from the Explosion}

After the player completes the "Plant the Storm Fuse", a timer for the bomb is set and the player must escape the level before the timer runs out. If the timer runs out, the level explodes and the player dies. Completing this objective means completing the level. 

\subsection{Progression Systems}

\subsubsection{Perks}

\subsubsection{Weapon Unlocks and Pick-ups}

\subsection{Play Flow}

Explain how the gameplay elements are connected to each other. Does it feel natural?

\subsection{Side-Challenge Structure}

The side-challenges exist to motivate the player to complete the level fully to achieve a higher rank, which is displayed on level completion. At the end of a level, the player is given information on their performance for each side-challenge. Completing all side-challenges should grant the player a reward that will help them on the journey, but I have yet to figure out if that should be perk-points or something else. 

\subsubsection{Kill All Enemies}

\subsubsection{Break your Evacuation Record}

\subsubsection{Take No Damage}

\subsection{Puzzle Structure}

Every level has one or two puzzles that are not so obvious to find. Each puzzle will require a unique weapon ability, or two, to complete. This puzzle structure's presence is similar to that of The Breath of The Wild. 

\section{Game Mechanics}

This is a critical part of the design document. Emphasize this section when writing.

\subsection{Rules}

\subsubsection{Health}

The player has a limited amount of health which, once depleted, kills the player. The health is divided into three segments. This means the player has, let's say, 120 health, but each segment is worth 40 health. When an enemy does damage to the player, they do 40 damage to deplete one segment of health. This exists so that the player can gradually heal when picking up scrap while maintaining a simple health structure. 

\subsubsection{Scrap Loss on Death}

When the player dies, they lose a certain number of scrap so that losing is discouraged.

\subsubsection{Rush Energy}

Rush Energy is a resource meter that allows the player to Power Slide, Recoil Boost and perform other Movement Perk abilities. Rush Energy gradually recharges when not used. 

\subsection{Interactions}

\subsubsection{Enemy vs. Player Contact}

If the player comes in contact with an enemy, the player will take a certain amount of damage

\subsubsection{Player Sliding into Enemies}

The player can slide into enemies to deal a small amount of damage and launch them up into the air. This is meant to let the player do one of two things: perform a "finishing move" to kill the enemy when the enemy's health is low, or to create crowd control. Additionally, this helps speedrunners complete ANY\% speedruns. 

\subsection{Physics}

The game contains standard 2D kinematic non-realistic physics. Refer to the different game mechanics to know the specifics.

\subsection{Economy}

Not strictly about currency, but about resource-management. For example, cooldowns on abilities or remaining items in inventory. 

\subsubsection{Scrap}

Scrap is a currency used call in supply drops that contain field- and standard weapons. Picking up scrap also heals you, but you store half  less scrap as you normally would when you have been damaged. 

\subsubsection{Field-Weapon Ammo and Cooldowns}

Field weapons have limited ammunition and some field weapon abilities are put on a cooldown when used. 

\subsection{Movement}

The player can move left and right just like any other 2D platformer. The walking is initially gradual but quick and seamless making it feel natural and easy to use. Jumping is also included in the character movement and the jump height is controllable.

\subsubsection{Slip Jump}

If the player jumps as soon as they change directions, the player receives a boost in momentum towards the new direction.

\subsubsection{Crouch and Sliding}

The player can crouch by pressing a button, making him slower, able to crawl under tight spaces and is harder to detect by enemies. When the player moves at full running speed on the ground, presses the crouch button and has enough Rush Energy, the player will slide on the ground with boosted momentum. The player cannot change movement direction when sliding on the ground, but can reduce the speed by attempting to change directions. The player can also slide as soon as they hit the ground, but the speed must be equal to or lower than the default running speed, otherwise the player will just continue to slip on the ground. Sliding can be used to keep as much momentum as possible when moving extremely fast since the player will experience less ground friction. 

\subsubsection{Power Slide}

Power Slide is a faster version of the standard slide that allows the player to ram into enemies to deal damage and knock them in the air. This is the default setting for sliding but require Rush Energy to activate. If the player has enough Rush Energy, they can Power Slide. Otherwise, sliding is only possible when the player already has momentum higher than the default running speed. 

\subsubsection{Recoil Boosting}

Firing a weapon while aiming downwards mid-air produces recoil that propels the player upwards so the player can stay in the air longer. Recoil Boosting depletes the Rush Energy meter. Every weapon propels a different amount and uses varying values of Rush Energy. 

\subsection{Objects}

\subsubsection{Scrap Collectible}

\subsubsection{Soul Core Collectible}

\subsubsection{Weapon Collectible}

\subsubsection{Incendiary Barrel Hazard}

\subsubsection{Spike Hazard}

\subsubsection{Null Cracked Wall}

\subsubsection{Metallic Surface}

\subsection{Combat}

Define the combat mechanics and their intended use against opponents and opponent moves. 

\subsection{Actions}

Explain all the possible actions the player is able to perform. This includes opening doors, picking up objects or attacks.

\subsection{Screen Flow}

Explain how the screen behaves when performing certain actions.

\subsection{Game Options that Affect Gameplay}

Describe all the possible options the player has and how the effect the gameplay further. 

\subsection{Replaying and Saving}

\subsection{Secrets and Easter Eggs}

\section{Story}

\subsection{Back Story}

\subsection{Plot Elements}

\subsection{Game Story Progression}

\subsection{Cut Scenes}

Descriptions about the actors included, the setting and a set of scripts.

\section{Game World}

\subsection{Look and Feel}

Define the visual and structural style of the game world. Is it a large island (open world)? Are there multiple worlds (Super Mario)?

\subsection{Areas (For Each)}

\subsubsection{Description and Physical Characteristics}

\subsubsection{Relations to the Rest of the World}

\section{Characters}

\subsection{Characters (For Each)}

\subsubsection{Back Story}

\subsubsection{Personality}

\subsubsection{Appearance}

\subsubsection{Abilities}

\subsubsection{Relevance to the Story}

\subsubsection{Relationships}

\subsection{AI and Enemies}

\subsection{Non-combat and Friendly Characters}

\section{Levels}

\subsection{Training Level}

\subsection{Levels (For Each)}

\subsubsection{Synopsis}

\subsubsection{Prerequisite Materials and How it's Provided}

\subsubsection{Objectives}

\subsubsection{Details of what Happens in the Level}

\paragraph{Map}

\paragraph{Critical Paths}

\paragraph{Important Encounters}

\section{Interface}

\subsection{Visual System}

\subsubsection{HUD}

\subsubsection{Menus}

\subsubsection{Camera Model}

\subsection{Control System}

\subsection{Audio}

\subsection{Art}

\subsection{Help System}

\end{document}