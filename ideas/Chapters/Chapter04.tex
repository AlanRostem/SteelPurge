\documentclass[../Main.tex]{subfiles}

\begin{document}


\section{Player Traits}

This section describes all the traits the player has regarding movement among other things.

\subsection{Hit Points}

The player can be hit by enemies or hazards up to three times before dying. In the lore, the player's suit can activate a protective shield every time they are hit. But, if they are hit all three times, it means the shield had no more energy left and thus the player died to the devastating damage. 


\subsection{Health regeneration and invincibility}

When the player is hit by an enemy, they are invulnerable for a short time. This will be displayed as a protective bubble around the player. This design's purpose is to tell the player that they can avoid damage for some time and use it to their speed-running advantage. Killing certain enemies can recharge your health, and this should be made apparent through the game art. 

\subsection{Chrono-drift}

Chrono-drift is an ability that lets the player travel back in time to a specific point quickly. The player must drop a pod on the ground for the ability to transport it back to that spot, but the pod itself does not exist for long. In the lore, the player is stabbing reality with the Chronodagger: the dagger that gives him this ability. 

\subsection{Movement}

The player movement in Steel Purge is a standard platformer movement with walking and jumping. Due to the limited input the player will have, there are some twists on how he can move.

\subsubsection{Gravity and Left-Right Movement + Jumping}

The player character in Steel Purge falls and can walk from left to right and jump on top of obstacles.

\subsubsection{Recoil-Hovering}

By firing downwards the player can fall slower after each fire interval. The player has a limited amount of recoil-hover shots which is indicated by a bar on top of the player's head. 

\subsubsection{Sliding}

The player can slide on the ground and on slopes. Sliding allows the player to build up momentum and jump in order to keep that momentum. Combining this with \emph{recoil hovering} lets the player keep the high momentum for longer. Sliding into enemies also deals damage to them and knocks them up into the air. 

\subsubsection{Ram-Sliding}

Sliding into enemies with high enough momentum can knock enemies into the air and deal damage to them. 

\subsubsection{Sliding-Jumping}

When the player slides and gains momentum, he can jump to avoid the ground-friction and keep the momentum to move fast for longer.

\end{document}