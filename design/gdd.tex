\documentclass[12pt]{article}

\title{Steel Purge Game Design Document}

\begin{document}

\maketitle{}

\section{Overview}


\subsection{Concept}

Smooth high mobility platformer game with various types of guns, levels, enemies and secrets. The game will feature multiple different objectives than the standard level completion of all platformers. 

\subsection{Genre(s)}

Run n' gun, platformer, sidescroller, action.

\subsection{Game Structure}

The game is a 2D sidescroller containing four levels per eight worlds where each level has a unique objective to complete per world. The objectives range from reaching the end of the level, collecting items and making sure you don't lose them, planting a bomb and last, but not least, defeating a boss.

% The game is a 2D sidescroller where the primary level type will start from left to right in the first half of the level objective like standard sidescrollers. The first half of the level is about carrying an explosive to a point. Along the way, there will be several obstacles and enemies posing a challenge to the player. After reaching the point, the player plant's the explosive and a timer starts. The player must now go back to the initial spawn point of the level within the bomb time limit. The level design will be two-way, meaning the first half will go from left to right and be more slow-paced as opposed to the second half which is more fast-paced. Since this is the primary level type, there will be more ways to play the levels, but this is the main selling point. 

\subsection{Look and Feel}

The game will be structurally similar to Super Mario Bros. on the NES and have 8bit style pixel art and music/audio with a mix of high-fidelity sounds. 

\section{Gameplay}

\subsection{Objectives}

\subsubsection{Purge the Steel}

This objective is the simplest of all. All the player must do is to exit the level in one piece. The level design will focus on aggressive combat play and will contain lots of secrets, but this does not need to be a strict rule. 

\subsubsection{Plant the Storm Bomb}

At the start of the level, the player is given an explosive item called the "Storm Bomb". The player must plant the item at the end of the level to complete half of the objective. After the player completes the "Plant the Storm Bomb", a timer for the bomb is set and the player must escape the level before the timer runs out. Escaping the level is done by going back to where the player started the level. If the timer runs out, the level explodes and the player dies. The level design will be created in such a way where it would be easy to learn going back and forth.

\subsubsection{Collect the Storm Cores}

The player must explore around the level to find Storm Cores, a collectible item, to progress towards completing the level. After collecting all Storm Cores, the exit will be unlocked and the player can complete the level by passing through the exit. If the player takes damage while having collected a Storm Core, they will drop it and must pick it up again. If the item falls into a pit or gets destroyed by a hazard, the item will spawn back at its original location and it will be indicated on the HUD. When holding one or more Storm Cores, taking damage will not drop your weapon. The level design will be very open, similar to metroidvania levels, and have the exit in an easily accessible location to make it more rewarding. 

\subsubsection{Defeat the Boss}

This type of level is the last level for the given world and the boss is located at the end of the level. All the player has to do is deal damage to the boss while dodging its attacks until the boss' health is depleted. This type of level will have the same design as the "Purge the Steel"-levels with an extra checkpoint and a boss-fight at the end. 

\subsection{Progression Systems}

\subsubsection{Scrap Score}

Scrap Score is a bragging-rights metric displayed at the title screen of the level showing how much scrap the player has collected in total after completing every level. 

\subsubsection{Field Weapons}

The only form for a progression system that technically counts as one would be picking up new weapons and learning how to use them. 

\subsection{Play Flow}

Explain how the gameplay elements are connected to each other. Does it feel natural?

\subsection{Secrets Structure}

Every level has one or two secret areas that are not so obvious to find. Each secret will require a unique weapon, or two, to complete. The secret areas will have a puzzle-like structure in the obstacles which require the specific weapons as well. The weapons required will correspond to the obstacles that require it as well. 

\subsubsection{Rewards}

Secrets can reward the player with two different things: a supply crate and a passage to skip some levels. The supply crates contain a large number of Normal Scrap and some ammo. Lastly, the secret passage leading to the next level world is an alternative to end the level that rewards the player with a new path to the next world inside the level-selection screen. This last part is ideal for speedrunning. 

\section{Game Mechanics}

\subsection{Rules}

\subsubsection{Taking Damage}

Taking damage depletes one dot of health and knocks the player backwards and sending them into a temporary invincibility state. Additionally, the player drops their weapon on the ground when taking damage. Then, the player is only given a small amount of time to pick up the weapon again before it automatically disappears. 

\subsubsection{Health and Death}

The player has a limited amount of health which, once depleted, kills the player and they must start over from their last checkpoint. Dying causes the player to lose scrap. The health pool is comprised of three dots. Taking any form for damage depletes one dot. Below the health bar, there is a bar that will fill up when collecting Health Scrap. Once the bar is full, the player will be given one dot of health. The bar cannot fill up when the player has full health, and enemies won't drop health scrap. Instead, the player will be given extra normal scrap. 

\subsubsection{Rush Energy}

Rush Energy is a resource that allows the player to Power Slide, Recoil Boost and Chrono Drift. The player has a maximum of six dots of energy displayed on their HUD and each dot refills slowly only when the player is grounded. Different weapons will use different amounts of dots per shot when recoil boosting, and power-sliding uses one. Chrono Drift must use two to balance such a powerful ability. 

\subsubsection{Scrap}

There are various types of scrap that enemies drop and they each have a different purpose. All types are dropped by enemies (some more/less than others) and are colored differently based on their use. All types of scrap spawn when an enemy is eliminated. Each is designed to reward the player for being aggressive, but also recover when they must play more carefully. 

\subsubsection{Scrap Loss upon Death}

The player will lose scrap each time they die they lose half of the scrap they have and it will be dropped in their death location to be picked up again. 

\subsubsection{Normal Scrap}

The first type of scrap is the "economical" scrap (just called Scrap), which is colored as a standard rusty cogwheel. This type of scrap's amount is shown in the HUD and the player can carry as much as they want of it (there should be a very high maximum due to technical reasons). This scrap can be used to refill health  and ammunition at checkpoints. % but can also be converted to Evo Cores when the player completes a level. For every hundred scrap, the player earns an Evo Core at the end of the level.
The only way to gain this type of scrap is by damaging an enemy, which makes weaker weapons useful for farming scrap. All scrap is lost when a level is completed and is converted to score. 

\subsubsection{Health Scrap}

Health Scrap is a green-colored cogwheel collectible that heals the player when collected. This type of scrap can only spawn when an enemy has below half of their health. After that, the health scrap can spawn when the enemy takes damage from the player. The companion drone starts using this scrap to repair the player's suit. 

\subsection{Interactions}

\subsubsection{Enemy vs. Player Contact}

If the player comes in contact with an enemy, the player will take a certain amount of damage

\subsubsection{Player Sliding into Enemies}

The player can slide into enemies to deal a small amount of damage and launch them up into the air. This is meant to let the player do one of two things: perform a "finishing move" to kill the enemy when the enemy's health is low, or to create crowd control. Additionally, this helps speedrunners complete ANY\% speedruns. 

\subsection{Physics}

The game contains standard 2D kinematic non-realistic physics. Refer to the different game mechanics to know the specifics.

\subsection{Economy}

Not strictly about currency, but about resource-management. For example, cooldowns on abilities or remaining items in inventory. 

\subsubsection{Field-Weapon Ammo}

Field weapons have limited ammunition. The player must conserve their ammo carefully. Each Field Weapon consumes their very own ammo pool which does not have a maximum. However, picking up a weapon initially gives a predefined amount. 

\subsubsection{Scrap}

Scrap is a gameplay currency used to refill ammo and health. When the player completes a level, they will lose all the scrap they collected. 

\subsection{Movement}

The player can move left and right just like any other 2D platformer. The walking is initially gradual but quick and seamless making it feel natural and easy to use. Jumping is also included in the character movement and the jump height is controllable.

\subsubsection{Slip Jump}

If the player jumps as soon as they change directions, the player receives a boost in momentum towards the new direction.

\subsubsection{Crouch and Sliding}

The player can crouch by pressing a button, making him slower, able to crawl under tight spaces and is harder to detect by enemies. When the player moves at full running speed on the ground, presses the crouch button and has enough Rush Energy, the player will slide on the ground with boosted momentum. The player cannot change movement direction when sliding on the ground, but can reduce the speed by attempting to change directions. The player can also slide as soon as they hit the ground, but the speed must be equal to or lower than the default running speed, otherwise the player will just continue to slip on the ground. Sliding can be used to keep as much momentum as possible when moving extremely fast since the player will experience less ground friction. 

\subsubsection{Power Slide}

Power Slide is a faster version of the standard slide that allows the player to ram into enemies to deal damage and knock them in the air. This is the default setting for sliding but require Rush Energy to activate. If the player has enough Rush Energy, they can Power Slide. Otherwise, sliding is only possible when the player already has momentum higher than the default running speed. 

\subsubsection{Recoil Boosting}

Firing a weapon while aiming downwards mid-air produces recoil that propels the player upwards so the player can stay in the air longer. Recoil Boosting depletes the Rush Energy meter. Every weapon propels a different amount and uses varying values of Rush Energy. 

\subsection{Collectible Objects}

\subsubsection{Scrap Collectible}

\subsubsection{Ammo Collectible}

\subsubsection{Weapon Collectible}

\subsection{Key Objects}

Key objects are objects that can trigger a special effect using a specific weapon's shots. They are meant to be used in combat and platforming.

\subsubsection{Incendiary Barrel Hazard}

The incendiary barrel is an explosive barrel that can only be triggered by shooting at it using flame-based weapons only. Enemies caught in the blast will be damaged, knocked away and have a burn-effect applied to them. If the player is caught in this blast, he will only take a small amount of damage and be knocked away far. The Incendiary Barrel can be knocked upwards using Power-Slide. The player can also push the Incendiary Barrel in two directions, but this slows down the player's movement and he cannot shoot his weapon. If shots that come from the Firewall or Falcon hit the Incendiary Barrel, then it will explode. It can also explode if it happens to collide with the tip of a Magma Spike. Explosions from other Incendiary Barrels can also trigger another Incendiary Barrel, but the KE-60's explosive projectiles can only push around the Incendiary Barrel, not detonate it. The explosions from Incendiary Barrels can also destroy breakable objects in the same way the KE-60's projectiles explosions can. These interactions exist so that the player can have many opportunities for completing mini-puzzles and take advantage of the Incendiary Barrel's properties during combat. 

\paragraph{Combat Purposes}

The player can use Power-Slide to re-position the barrel quickly to deal massive damage to an enemy elevated higher. The primary usage for this object is to explode it using a flame-based weapons to clear out many enemies nearby or weaken strong enemies such as mini-bosses. This can also give the player momentum to escape combat at a cost for health. The player just needs to jump above the barrel, aim down and shoot at it with their flame-based weapon to use the explosion to launch themselves far away from combat. This is risky, but can help the player complete the level faster. 

\paragraph{Secret Purposes}

This paragraph will not describe in detail how the Incendiary Barrels will work in puzzles. However, the Incendiary Barrel's properties open up for a lot of opportunities in creating puzzle structures. The main thing the Incendiary Barrel provides is the possibility to push it towards a breakable structure that lets the player access a secret area behind it by exploding it. 

\subsubsection{Magma Spike Hazard}

The Magma Spike is a breakable object that can be destroyed using explosions. If an enemy or the player hits it, they instantly die. 

\paragraph{Combat Purposes}

There is not much talk about when it comes to combat. It can help the player escape when destroyed and using utilities that push enemies towards it can also be helpful since it instantly kills them. The KE-60 is a perfectly balanced example of this, since the player can use that weapon to knock enemies into the spikes, but also risk destroying it in the process, making it a strategic decision.

\paragraph{Secret Purposes}

The Magma Spike hazard prevents the player from reaching an area since it instantly kills him, making it easy for the level designer to hide secret areas since it is not apparent to the player that they can access what's behind the spikes. Accessing what's behind the spikes is only possible using explosions, which the KE-60 weapon easily provides, but it's also slightly possible using flame-based weapons too. 

\subsubsection{Null Cracked Wall}

\subsubsection{Metallic Surface}

\subsubsection{Null Transporter}

\subsection{Other Objects}

\subsubsection{Storm Fuse}

\subsubsection{Crate}

Crates are breakable objects that grant a reward for the player. Crates can contain different types of ammo and scrap. Some rarer crates also contain a weapon, but the weapon itself is not revealed until the crate is broken. The containments will be indicated by an icon on the crate. Power-slide, shooting or any other damaging ability can crack open a crate and drop its items.  Power-sliding into a crate causes the contents to fly up into the air. 

\subsubsection{Damaged Obstacle}

Damaged obstacles are destructible and grant scrap upon taking damage and when destroyed, similar to how enemies work. They are used to add a fun way to progress the level and reward some scrap. These are square-shaped and blocky, which is meant to be distinguishable.

\subsection{Enemy Mechanics}

This section focuses on the enemies and their varying mechanics described in a similar fashion to the Objects section. 

\subsubsection{Melee Hitbox}

All enemies have a default hitbox that damages the player with one bar of health when the player touches the hitbox. 

\subsubsection{Damage Reception Hitbox}

This hitbox is the default area of the enemy that the player must deal damage to using weapons or abilities. The hitbox spans the entire body of the enemy.

\subsubsection{Critical Damage Reception Hitbox}

This hitbox is similar to the default damage reception hitbox, only that it applies more damage to the enemy than normal. Additionally, this hitbox rests on top of the enemy's head and can only be hit if the player is above the enemy and aiming downwards at its head. Critical hit multipliers will varying amongst each enemy and some may die in one hit or more. 

\subsubsection{Types of Enemies}

\paragraph{Normal Enemy}

Normal Enemies are the minions that the player can kill without hitting too many shots. These are the most common and frequent enemy type the player will encounter. 

\paragraph{Gunner Enemy}

The Gunner Enemies are humanoid-looking enemies that drop Field Weapons upon death. Their combat design will vary depending on the weapon they carry, but the visual design is just a re-skin of a default base. If the player already has unlocked the weapon they carry, the enemy will just drop a high amount of ammo for that weapon (indicated by their element). Gunner enemies are a type of mini-boss that will be very agile and somewhat resemble the capabilities of the player. Gunner enemies only fire their weapon in a certain interval followed up by a offensive/defensive usage of the weapon's special ability. These enemies are supposed to be harder to defeat than normal enemies (both by their AI and HP), but easy for experienced players (by learning the enemies). These enemies are not common, but start becoming more common after the first encounter since the player would have unlocked the enemy's weapon by the time it was defeated. 

\paragraph{Mini Boss}

\paragraph{Boss}

\subsubsection{Elements of Enemies}

The enemies will have an element signifying what type of ammo they will drop on death. Not all enemies will have an element tied to them because that would decrease the variety in their behaviour for the following reason: elements also signify their battle-style. For instance, Fire-based enemies will fight close-range and deal high damage per hit. The element will be displayed on the enemy based on their visual design using colors. Enemies without an element will not drop ammo and only scraps and will have their own unique way of fighting. Mini-bosses and bosses are always going to have an element tied to them, though some exceptions can be made depending on level design. 

\paragraph{Element Playstyles}

\begin{itemize}
	\item Flame: Operates in close-range combat and deals high damage
	\item Energy: Durable heavy enemies that deal concentrated damage focusing on locking down an area
	\item Dark: Long-range operating enemies with evasive abilities
\end{itemize}


\subsection{Enemy List}

Here is a list of all the enemies and their different mechanics. 

\subsubsection{Enemies (For Each)}

\paragraph{Moves}

\paragraph{Durability}

\paragraph{Mobility}

\paragraph{AI}

\paragraph{Difficulty}

\paragraph{Encounters}

\subsubsection{MXD-Hornet}

\paragraph{Moves}

The only move this enemy has is rushing towards the player.

\paragraph{Durability}

This enemy is is not very durable and critical hits from Field Weapons are a one-hit kill.

\paragraph{Mobility}

This enemy is very fast and can often catch the player off-guard. 

\paragraph{AI}

The MXD-Hornet is a spider-like enemy that moves very fast on the ground. At first, it patrols from left to right and when the player is near it, the enemy will briefly stop and start to rush towards the player. If it hits the player, the enemy explodes and does damage to the player. If the player jumped over the enemy during the rush, the enemy will stop for a moment do it again. 

\paragraph{Difficulty}

This enemy is meant to be very easy to defeat especially with the core mechanics of Recoil-Boosting

\paragraph{Encounters}

This enemy is very common and can be seen often throughout the first levels.

\subsubsection{Armoured MXD-Hornet}

This enemy is the exact same as the regular MXD-Hornet, but it can only take critical hit damage. The Firewall .127 upgrade and explosions are the only ways to directly damage this enemy. 

\subsubsection{Miner Exo}

\paragraph{Moves}

The only move this enemy has is striking the player with its pickaxe.

\paragraph{Durability}

This enemy is slightly more durable than the MXD-Hornet for reference and requires a few more critical hits to kill. 

\paragraph{Mobility}

The enemy can only walk normally towards the player, and is much slower than the MXD-Hornet for reference.

\paragraph{AI}

The enemy's default behaviour is to walk back and forth until the player is within a certain distance. Once the player is close enough, the enemy will start to charge at the player with its pickaxe and attempt to strike the player.

\paragraph{Difficulty}

This enemy is meant to be very easy to defeat especially with the core mechanics of shooting and evading attacks. 

\paragraph{Encounters}

This enemy is very common and can be seen often throughout the first levels.

\subsubsection{Ranged Miner Exo}

Ranged Miner Exo is the same as the regular Miner Exo, but the AI is a bit different. It will try to keep distance instead of charging towards the player and throws two pickaxes each time it attacks instead of meleeing. 

\subsubsection{Firewall Gunner Exo}

\paragraph{Moves}

Like all Gunner Exos, the Firewall Gunner will only have two attack moves: shoot and special ability. The first move is to fire the equipped weapon in a certain interval, and the second move is to unleash the deadly fire stream from the weapon.

\paragraph{Durability}

This enemy is as durable as any Gunner Exo. Gunner Exos have a higher durability than normal enemies and tend to be as strong as mini-bosses. 

\paragraph{Mobility}

This enemy has the same base mobility as any Gunner Exo. It is able to move left and right with faster movement than the normal enemy.

\paragraph{AI}

This enemy will move left and right while simultaneously firing 2-4 shots out of its weapon no matter how close the player is to it. These intervals are fixed and the player can predict them. The enemy is also able to shoot upwards if the managed to jump above it during the shooting interval. After this shooting interval, the enemy will try to retreat and use the fire stream ability, but it will only be fired horizontally, giving the player the chance to land several critical hits. 

\paragraph{Difficulty}

This enemy is meant to be slightly harder than normal enemies to defeat, but also very easy if the player has learnt the enemy's AI. 

\paragraph{Encounters}

This enemy will have the same placement in a level like any other Gunner Exo. The enemy will be rarer than the normal enemies and won't be encountered very often until the player has already encountered at least one of them (to unlock the weapon they wield). 

\subsection{Combat}

The combat mechanics in this game are very simple. The player can Power Slide, jump and fire their weapon. Lastly, enemies have a weak spot on top of their head. The player must jump over the enemy and aim down to be able to hit it. Most enemies die from one hit on this weak spot and some take a few more hits. Power-sliding knocks enemies up in the air and leaves them vulnerable to the player's shots from below since some enemies cannot react to the knock-up.
These are the core combat mechanics, the rest is determined by the weapon the player has equipped.

\subsubsection{Weapon Switching}

The player can switch between all his unlocked Field Weapons and equipped Standard Weapon. Switching weapons causes a delay before the player can use it. This lets the player strategically choose which weapon to conserve ammo for and not be completely reliant on their Standard Weapon. 

\subsubsection{Playstyles tied to Weapon Element}

As described in the section about elements for enemies, the same logic applies to the player with their equipped weapon. Flame-based  weapons are effective at close range, Energy-based weapons are easier to use when standing still and Dark-based weapons fire slowly so the player is safer at longer distances. These playstyles can change based on what the player wants to do, and practically all weapons are effective in all capacities if the player is good enough at the game. 

\subsubsection{Field Weapons}

Field weapons are weapons that can only be unlocked by picking it up for the first time. Field weapons come with limited ammo in specific amount, but the player can pick up as much ammo as they want for a weapon. The player carries the weapon they picked up to the next level. 

\paragraph{Neostar Machinegun}

Rapid-fire magnetic machine-gun. A quick charge up time is required for the weapon to fire, but it can be spammed to keep it at almost-max charge. Hitting an enemy enough times magnetizes the enemy and makes all your shots follow the enemy. It is also possible to magnetize metallic surfaces in the same way.

\begin{itemize}
	\item Initial Ammo: 200
	\item Damage: 2
	\item Recoil-boost usage: ??
\end{itemize}


\paragraph{KE-60 Grenadier}

Explosive energy orb launcher that can propel any object, including the player. The propelling property of the orbs is meant to allow the player to rocket-jump and gain massive momentum at a cost for HP since it also damages the player. The explosion from the orbs can also destroy stone-based hazards and obstacles. 

\begin{itemize}
	\item Initial Ammo: 30
	\item Direct-hit Damage: 5
	\item Explosive Damage: 5
	\item Recoil-boost usage: ??
\end{itemize} 


\paragraph{Firewall Shotgun}

Flame shotgun that burns enemies over time. The weapon fires a spread of flame pellets for each shot. This is also the only weapon that can trigger explosive barrels, which do high damage to enemies and allows the player to rocket-jump on them. 

\begin{itemize}
	\item Initial Ammo: 50
	\item Damage per Pellet: 2
	\item Pellet count per shot: 5
	\item Burn-damage per second: 2
	\item Recoil-boost usage: ??
\end{itemize} 

\paragraph{Falcon Rifle}

Rifle with a rocket and bayonet attached. It shoots projectile bullets propelled by the rocket receiver of the weapon. Wielding this weapon allows the player to slide faster and deal more melee-based damage (i.e., Power Slide). If the player runs out of Rush Energy, they can still Power Slide with the default speed and damage, but this consumes ammo. 

\begin{itemize}
	\item Initial Ammo: 50
	\item Damage: 5
	\item Recoil-boost usage: ??
	\item Slide ammo consumption: 5
\end{itemize} 


\paragraph{Nullshatter Sniper}

Reality-shattering inter-dimensional sniper rifle. Fires slowly and makes infinitely long cracks in space-time to deal damage to anything that comes in contact with it. This weapon generally instantly kills enemies, but has a long delay after each shot. Shooting at Null Transporters activates a portal that you can use to briefly enter the Null Dimension and avoid all damage. Being in the Null-Dimension opens up secret paths the player can take.

\begin{itemize}
	\item Initial Ammo: 30
	\item Damage: 
	\item Recoil-boost usage:
\end{itemize}

\paragraph{Stormhammer}

\begin{itemize}
	\item Ammo capacity: 100%
	\item Damage: 
	\item Recoil-boost usage:
\end{itemize}

\subsubsection{Standard Weapon}

The player is equipped with a short-ranged pistol that deals a small amount of damage but shoots infinitely. It has no special trait.

\paragraph{Hamilton M99 Pistol} 

Default pistol with infinite ammunition that fires rapidly. The damage is hit-scan based. It lets the player Recoil-boost in the air without falling.

\begin{itemize}
	\item Ammo: Infinite
	\item Shot type: Limited range hit-scan
	\item Damage: 1
	\item Recoil-boost usage: ??
\end{itemize}

\subsection{Actions}

Explain all the possible actions the player is able to perform. This includes opening doors, picking up objects or attacks.

\subsubsection{Chrono Drift}

The companion drone that stands alongside the player has the ability to travel back in time at any moment. With the press of a button, the player can save their current location and situation for later time travel. Double-tapping the button will reset time back to the point the player saved. The reason for double tapping is so that the player does not accidentally travel time after making lots of progress. Pressing the button once just saves time in the player's current location again. This ability is in place so that the player can learn the levels without having to go back too much, but this must be a concious decision. All of the player's stats (such as health and weapon ammo) will remain the same when travelling time, but every other object in the level will reset. However, the scrap count will reset to balance this effect since the player must not be able to infinitely generate scrap. At the end of a level, it shows how much the player used this ability and the player will be rewarded with a higher rank when they don't use it at all throughout the level. Chrono Drift usage drains Rush Energy to limit the player's ability to exploit it.

\subsection{Screen Flow}

Explain how the screen behaves when performing certain actions.

\subsection{Game Options that Affect Gameplay}

\subsubsection{Weapon Switching}

Picking up a different weapon can limit or open up the possibilities the player has a ahead. This includes accessing key objects, affecting mobility and affecting the ease of defeating a specific type of enemy. 

\subsection{Replaying and Saving}

\subsection{Secrets and Easter Eggs}

\section{Story}

\subsection{Back Story}

The back story will be told throughout the main story of the game. 

\subsubsection{A New Home}

It is year 3036 and the Earth is reaching its inevitable demise from the global warming. Scientists across the globe banded together to locate a habitable planet to transfer the human race to a new home. Planet HS-921 was a great candidate for a new home since it was several times larger than Earth and distanced perfectly from its sun to host life on its surface. In this time period, the human race was  developed enough to travel across solar systems, but not everyone could join the journey. Only rich people, scientists and military personnel were privileged to live on HS-921. The global warming apocalypse had deprived resources for building enough spaceships for the whole population, but the scientists promised to use HS-921's resource-rich ground for building an opportunity for the people back home to travel as well. Sadly, several decades passed by as the earthlings were being consumed by the apocalypse. The inhabitants of HS-921 never came back to help the earthlings, but it was never their intention to begin with. This promise was only used as propaganda to let rich people and scientists escape for a better life and leave Earth to rot alone. 

\subsubsection{The Exo}

In this time of Earth's abandonment, the people of HS-921 thrived upon Dr. Hamilton's Fabricator. The mysterious and exceptional scientist developed a machine that extracts metallic minerals from HS-921's core while simultaneously building and programming helper robots that did all the manual labour required to build a home. These robots were called the Exo, a highly modular and intelligent manual labour robot serving the creation and maintenance of a new planetary home. The Fabricator generated more Exos than the human population on HS-921. This meant both a good and a bad thing...

\subsubsection{The Shock}

Several more decades passed with the Exos' helping the people on HS-921 build a home, until a sudden electric blast hit the atmosphere. This shock briefly disabled all computer technology, including the Exos. A few days pass by without any technological activity, making it difficult for people to develop the new home, let alone survive in it. After a crippled development, the Exos started to activate again, but they turned hostile towards the people on HS-921. Everyone was attacked and imprisoned while many Exos were taking over every important facility and Fabricator. The people were hopeless to prevail, since not even the military personnel had functional weapons after The Shock. 

\subsubsection{A Spark of Hope}

Dr. Hamilton, despite the technological cripples, managed to develop a replacement for computer-based technology. Planet HS-921 happened to generate a strange and mythical power through its storms. Hamilton  discovered a crystal that can call in the mythical storms. These crystals are called Storm Shards and were used to create Storm Cores: a device that manipulates The Storm's energy. This energy was be used to develop computer technology that was not susceptible to the same Shock the Exos and other standard computer technology was. New and more powerful military weapons were deployed and communication systems could never be compromised ever again. There was hope for survival, but The Storm had a limit... Utilizing a high amount of Storm Energy per Storm Core would deplete the Storm Shards material stability, rendering it useless. Thus, a group of highly skilled soldiers were trained to utilize The Storm's power to fight back against the Exos in an efficient manner. These highly formidable soldiers were known as the Turbo Troopers. They were trained to be fast, agile and smart with their resources at hand. They were equipped with weaponry that could shoot as long as they do not use too much energy and suits that increased their movement speed, protected them from heavy damage and made them physically stronger. 

\subsubsection{The Great Metal War}

It was the Turbo Troopers against the Exos. A war broke out between a limited amount of highly skilled soldiers and dangerous self-developing robots. For every battle won by the Turbo Troopers, they collected the scrap metals from the defeated Exos for later use. Sadly, the Exos outnumbered the Turbo Troopers and more died as the Exos produced more of themselves using the hacked Fabricators. There were less and less Turbo Troopers as days went on. Dr. Hamilton's efforts were exceptional and delivered hope, but one man cannot stand against an infinite army of lethal robots. A few Turbo recruits were left alive since they were never sent out to battle, and luckily, there was still hope of finding the last professional Turbo Trooper...

\subsubsection{The Last Trooper}

In the early days of The Great Metal War, Dr. Hamilton secretly instructed a specific Turbo Trooper to deliver Storm Shards to Earth to save the people with a new technology. Despite the lies told to the earthlings, Hamilton was one of the only scientists that wanted to save Earth, he just needed time to figure out a solution and deliver it. Sadly, this Turbo Trooper couldn't complete the mission... Flying in his spaceship, right before reaching the first layer of HS-921's atmosphere, The Storm struck the Trooper's spaceship and caused it to crash-land through a mountain and into a mining facility inside it. Strangely, the ship was covered in Storm Energy and protected it from the crash damage, but the Trooper piloting it suffered a concussion and was laying unconscious for several days. During this time, the war was still ongoing. The Turbo Trooper was laying unconscious until the Exos won and the people of HS-921 lived in constant oppression...

\subsubsection{HM-4170's mission}

Dr. Hamilton realized that there is barely any hope of training Turbo Troopers. He developed a survey drone with the power to travel time, but this time power had some limits. This drone could only travel time to observe a location's past, but in the event that the drone succeeded the mission, the Trooper would be granted to use this power in dire needs. HM-4170, the survey drone, went to scout for the Turbo Trooper. The drone analysed as many locations as possible and managed to locate the crash site. The Turbo Trooper was found laying in an electrified ship and completely unconscious. HM-4170 depleted the Storm Energy surrounding the crashed ship and revived the Trooper. After reviving him, the drone introduced itself, but it appeared that the Trooper had lost all his memories from the concussion. HM-4170 instructed the Turbo Trooper to ready his weapon and get out of the mining facility before being consumed by the Exos ahead. The game begins... 

\subsection{Plot Elements}

\subsection{Game Story Progression}

\subsection{Cut Scenes}

Descriptions about the actors included, the setting and a set of scripts.

\section{Game World}

\subsection{Look and Feel}

The game world is in a sci-fi-style planet with life similar to nature on Earth but with alien-like quirks. The game will feature a level selection screen from a top-down world inside a home base where the player selects where to land. The game is structured from a level-selection screen where the player chooses which level to play on. Each level is a sidescroller 2D level similar to Super Mario games. 

\subsection{Areas (For Each)}

\subsubsection{Description and Physical Characteristics}

\subsubsection{Relations to the Rest of the World}

\subsection{Mining Facility FB-12}

This area is where the first level will be a part of, but as soon as the player completes this level, other areas will be used for the further levels. This area is still possible to come back to for extra content. 

\subsubsection{Description and Physical Characteristics}

This area is a usual mining facility inside a large mountain cave, but with lost of advanced metallic sci-fi mineshafts and storage rooms. The background of the levels will feature stalagmites and gemstones with some metallic structures holding things in place. Some areas will have piles of stones and ores to signify that the place is under work by the Exos. Some places will also have Exos mining ores with pickaxes. There will also be some incendiary barrels that contain gunpowder which can be exploded using the Firewall .127 weapon. 

\subsubsection{Relations to the Rest of the World}

This mining facility is one of many facilities used by the people of HS-921 to gather resources. The Exos being created from the Fabricator located here are programmed to adapt to the harsh conditions of mines and learn efficient ways to mine resources. Just like every other Exo, these were hacked to attack any human in sight and will use all their tools at their disposal to attack the main character. This is where the main character crash-landed and was found by HM-4170.

\section{Characters}

\subsection{Characters (For Each)}

\subsubsection{Back Story}

\subsubsection{Personality}

\subsubsection{Appearance}

\subsubsection{Abilities}

\subsubsection{Relevance to the Story}

\subsubsection{Relationships}

\subsection{HM-4170}

\subsubsection{Back Story}

HM-4170 is a Chrono-Drifted Survey Drone. It was created to find the last surviving Turbo Trooper after The Great Metal War. HM-4170's protocol is to never interact with any object while back in time and to only scan it to gather information.

\subsubsection{Personality}

Inspired by similar robot characters such as BT-7274 from Titanfall 2. HM-4170 is innocent and obedient with a problem-solving and supportive initiative. He does not have the best social skills and lacks the ability to understand sarcasm and irony, but is always willing to learn.

\subsubsection{Appearance}

\subsubsection{Abilities}

HM-4170 can hover and travel back in time. %, but he also has the ability to use Storm Cores to give the player a power-up. 

\subsubsection{Relevance to the Story}

\subsubsection{Relationships}

\subsection{AI and Enemies}

\subsection{Non-combat and Friendly Characters}

\section{Levels}

\subsection{General Structure}

\subsubsection{Combat Areas with Mini-Objectives}

Combat areas with mini-objectives are separate areas on the levels where certain enemies appear to guard the player against an objective. The types of enemies and amount of each type are placed in a certain way to have synergy so that the combat does not feel shallow and too simple. These combat areas are not meant to be the primary way to encounter and fight enemies since enemies will appear almost anywhere depending on a given level's design. 

\subsubsection{General Areas (Mixed with platforming and combat and secrets)}

WIP: This subsection describes the general areas the player will encounter as a continuous pace of the level. Since this game is a sidescrolling platformer, there will obviously be platforming on the levels. This includes simple parkour by jumping and using other movement mechanics and avoiding hazards. From the section describing all the game objects, some objects will be used to help the player either in combat or to complete mini-puzzles to unlock secrets. Some objects can be interacted with only by shooting at it using a specific weapon and the rest have their own specific quirks. The objects will be based on the level's theme, which is commonly tied to a field weapon (also ones that are meant to be unlocked on the level itself). Generally, there will always be a small amount of enemies and environmental hazards for the player to encounter and overcome as part of the level challenges. Using the objects that can only be interacted with using Field Weapons, the player can access secret areas and get rewards. The objects will be placed in a non-obvious manner. 

\subsubsection{Secret Areas and Rewards}

Secret areas can be reached through plaforming or using a weapon's unique trait to access a key object to help with the platforming. Secret areas can reward the player with weapon ammo, scrap or sometimes the ability to skip many levels. 

% Secret areas can be reached through puzzles using the companion drone, regular exploration, platforming and using Key Objects. Secrets can reward the player with a High-Tier Power-Up or a Weapon Upgrade.

\subsubsection{Checkpoint Placement, Usage and Purpose}

Despite the fact that the player can make a checkpoint wherever they want using Chrono Drift, there will be "automatic" checkpoints placed throughout the level with extra functionality. These checkpoints are where the player will respawn if they die, hence why it was referred to as an automatic checkpoint. The player just needs to collide with the checkpoint to mark it as a respawn-point. Additionally, this type of checkpoint allows the player to use scrap to instantly refill ammo and heal up to full health. The checkpoints are made to get the player up-to-speed on the challenges that are to come and make it more intuitive to retry after death. They are normally placed so that they are available after several combat areas and before objectives, puzzles and platforming areas. However, this can vary depending on a given level's design. Levels will only have one checkpoint to make the game simple, but the exception is levels with boss fights that always have a checkpoint near them (which makes the level have two checkpoints). 

\subsubsection{Objective Placement and Types}

\paragraph{Planting Bomb at Objective}

This objective type will be most common for the levels. At the very end of a level, there is a large Fabricator that the player must plant an explosive at (simply by pressing a button). This starts a timer before the explosive detonates so that the player has room to escape the level before the detonation. At the objective point, there will be enemies guarding it. Depending on the level's design and difficulty, there will be either regular enemies (small minions) or mini-bosses guarding it, so the player must defeat them before proceeding. Escaping the level successfully (returning to the start of the level) within the timer means completing the level. 

\paragraph{Boss Fights}

Some levels have a boss at the very end which the player must defeat in order to complete the level. Each boss design will be unique and have a relevance to the story. 

\subsection{Training Level}

\subsection{Levels (For Each)}

\subsubsection{Synopsis}

\subsubsection{Prerequisite Materials and How it's Provided}

\subsubsection{Objectives}

\subsubsection{Details of what Happens in the Level}

\paragraph{Map}

\paragraph{Critical Paths}

\paragraph{Important Encounters}

\section{Interface}

\subsection{Visual System}

The interface will be a simple pixel-art based interface with menus and a HUD. Progress bars will be avoided as much as possible for conveying gameplay elements since it is too difficult to do in a pixel art style. 

\subsubsection{HUD}

The in-game heads-up display will only show necessary information about the player's status initially. This includes health, equipped weapon, ammo scrap. Some weapons may have additional user interface elements to display the availability of certain actions (i.e., charge-up time). Lastly, the Rush Energy meter is also a part of the HUD, kind of. It is only displayed when drained or recharging and hovers over the player's head inside the game world. 

\begin{itemize}
	\item TODO: Show images of all the HUD behavior
\end{itemize}

\subsubsection{Gameplay Dialogues}

Characters that speak to the player through the story will have their dialogues conveyed through text boxes that pop up on the player's HUD. The companion drone will often talk to the player, so that is why dialogue boxes will often appear while the player is playing in a level. Dialogue boxes appear at the top of the screen with an image of the character next to its name. The text continuously writes itself in a slow manner, but the player can press and hold the interact button to make it go quicker. After the text is complete, the player can press the interact button to move to the next line. The button bound to skipping lines will also show up next to the dialogue. Dialogue boxes can also be disabled in the settings menu. 

\begin{itemize}
	\item TODO: Show images of all the dialogue behavior
\end{itemize}

\subsubsection{Companion Drone Speech Bubbles}

The companion drone will sometimes assist the player using hints. This is conveyed through speech bubbles above the player (since the drone is rested on top of the player's head). Hints are short and are usually used to remind the player about the mechanics in the beginning. However, further down the game, these will be used for dramatic effect through the story.

\subsubsection{Menus}

\subsubsection{Camera Model}

\subsection{Control System}

This section focuses on the varying controls the player has available and the scope of them. 

\subsubsection{In-Game Input}

The input options when the player is playing in a level will be limited to the buttons on a classic SNES controller. That does not mean other controllers are not compatible, but that the button layout is limited for the simplicity of the game. Even a keyboard and mouse will be usable, the point is that there won't be the need for more buttons than that of a SNES controller. 

\begin{center}
\begin{tabular}{ | l | l | }
 \hline
 \textbf{Input option} & \textbf{Action} \\
 \hline
 R & Attack \\  
 \hline 
 L & - \\
 \hline
 D-pad Left & Walk Left \\
 \hline
 D-pad Right & Walk Right \\
 \hline
 D-pad Up & Aim Up \\
 \hline
 D-pad Down & Aim Down (mid-air only) \\
 \hline
 D-pad left & Walk Left \\
 \hline
 Hold A & Crouch/Slide \\
 \hline
 B & Jump \\
 \hline
 Hold B & Control Jump-Height\\
 \hline
 Y & Set Time Travel Point \\
 \hline
 Double-Tap Y & Travel Time \\
 \hline
  Y (When near object) & Interact \\
 \hline
 X & Switch Weapon \\
 \hline
 Start & Pause Menu \\
 \hline
 Select & TBD \\
 \hline
\end{tabular}
\end{center}

\subsection{Audio}

The audio will include a mix of 8-bit sounds and a very small amount of semi-realistic sounds. The same will apply to the music, where real instruments and 8-bit music is mixed together, but 8-bit will still dominate. This is to make it easier to implement audio/music, since it breaks the limitation of 8-bit and adds a little charm to the musical art. 

\subsection{Art}

The game's art will be pixel-art with a light-hearted science fiction look using a limited color palette. The levels will be structured by pixel-art 16x16 tiles and most character sprites will have 32x32 resolution. There will be several different environments, but the general look of the environment will have an alien-like nature mixed with sci-fi buildings. 

\subsection{Help System}

\end{document}